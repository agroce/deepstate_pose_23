 %%%%%%%%%%%%%%%%%%%%%%%%%%%%%%%%%%%%%%%%%%%%%%%%%%%%%%%%%%%%%
\documentclass[12pt]{article}

\usepackage{fullpage}

\begin{document}

\pagenumbering{gobble}

\begin{center}
  {\Large\sf\textbf{POSE: Phase II: DeepState: Property-Driven Generalized Unit Fuzzing and Symbolic Execution for C and C++ Code}}
\end{center}

The data in the proposed project is primarily source code for the open
source project to be developed.  This project has an existing
repository that contains the existing source code, to which the PI has
full access.

Documentation and curricular items associated with any tools will also be stored in
an open source repository, since in this project the primary
educational benefits are linked to the use of these tools.  Using GitHub automatically provides us with excellent backup
and archiving for the code and curricular material products of the
project.

We have no unusual format or metadata requirements; the
tools involved primarily use textual formats that are easy to store
and read or byte-based formats that are interpreted only by fuzzer
drivers or code under test.

\textbf{Human Subjects Data and Access}

Data to be gathered or produced in this project will include data from user studies including field notes, digital audio and video 
recordings from interviews and surveys with human subjects, and transcriptions of interview results. Additional metadata will include 
coding schemes and memos, experiment protocols, design recommendations, and derivative ideas for improving the community. This data 
will be suitably anonymized, such that participant location or identity is never recorded. Human subjects' raw data will be kept secure 
in accordance with approved IRB protocols and best practices for protection of human subjects.

\textbf{Data Archival}

Project results will be archived at each university on their local servers, managed and supported by their department IT team. Individual 
human subjects’ data will be archived on a password-protected system accessible to only the PIs and their research team. Raw data will 
be archived for a limited period of time of no more than five years from the completion of the project as required by the human subjects 
protocol; after this period of time has expired, the data will be destroyed to protect the privacy of individuals involved.


\end{document}
