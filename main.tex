\documentclass[numbers]{proposalnsf}

\title{POSE: Phase II: Practical Extensible Mutation Generation and Management for Any Language}
\author{Alex Groce }
\date{August 2023}

\newcommand{\um}{\texttt{universalmutator}}

\begin{document}

\section*{POSE: Phase II: Practical Extensible Mutation Generation and Management for Any Language}

\subsection*{Overview}
\vspace{-2mm}

However, the growth of \um\ is limited by (1) its flat/informal leadership structure with only one author at the top who can approve new code contribution and (2) lack of documentation and support for development environment integeration, and (3) lack of external contributors from various associated communities.
We propose to expand the open-source ecosystem of users, contributors, and developers of \um, by fixing these problems.
In particular, we propose to work on creating (1) a written governance document with a new hierarchical leadership structure, (2) new documentation materials for onboarding new users/contributors, including translations and community standards to encourage diversity/inclusion, and (3) infrastructure for connecting \um\ to other tools.
Furthermore, we plan to systematically evaluate the effects of our project on the \um\ ecosystem by measuring changes to important metrics (number of unique contributors, diversity of contributors. 
The result of this project will be a self-sustaining open-source ecosystem for \um, which will allow it to grow into a locus for the widespread practical application of (and research on) mutation testing.

\subsection*{Intellectual Merit.} 
\vspace{-2mm}
Modern software engineering for critical systems relies on automated tests to encode and enforce notions of correctness.  Determining whether such tests are sufficiently powerful to detect anomalous behavior is a fundamental problem of computer science; mutation testing, thanks to advances in computing power, has become the most promising practical method for this task.  This project will create new documentation expanding the \um\ ecosystem, making it possible for developers to apply mutation testing in novel languages or using project-specific operators, and enabling the development of novel tooling for novel applications of mutants to software engineering tasks.  More researchers and developers will be able to use \um\ to explore and enhance the power of mutants to guide software development.

\subsection*{Broader Impacts}
\vspace{-2mm}

Efficient software is increasingly important in our modern digital/networked society since ``software is eating the world'' and this process shows no sign of stopping or even decelerating. 
Our project will benefit every area of society where software is employed since our project aims to improve a fundamental tool for effective software testing, \um..
Furthermore, because our project emphasizes documentation and translations for onboarding new users/contributors, our project will result in more developers, and more diverse developers, becoming aware of this powerful tool for mutation-based analyses of code and tests.

\paragraph{Keywords:}
CISE; mutation testing; software testing infrastructure; software specification

\pagenumbering{gobble}
\newpage  
%\pagenumbering{arabic}
\pagenumbering{gobble}
\section{Context of the Open-Source Ecosystem}%section required by call, otherwise RWR

\end{document}